%!TEX root = ../dissertation.tex

\begin{savequote}[75mm]
Everything exposed by the light becomes visible, for everything that becomes visible is light
\qauthor{Ephesians 5:13--14}
\end{savequote}


\chapter[short-title]{Full but might be too long title}

\lettrine[lines=3]{\textcolor{SchoolColor}{T}}{raditional work on intrinsic image
decomposition} relying on physical characteristics produce high qualitative images,
while deep-learning-based models dominate quantitative results.
In this chapter, we propose a deep-learning-empowered method steered by the
physics-based reflection models, thus achieving the best of the two worlds.
The network architecture, coined RetiNet, exploits reflectance and shading
gradients to obtain intrinsic images as inspired by the well-established
Retinex model. The proposed approach allows for the integration of all
intrinsic components. To train the new model, an object centered large-scale
datasets with intrinsic ground-truth images are created.
The experimental evaluations show that the new model outperforms existing
methods. Visual inspection shows that the image formation loss function augments
color reproduction and the use of gradient information produces sharper edges.

\section{Introduction}

